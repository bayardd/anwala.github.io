%%%%%%%%%%%%%%%%%%%%%%%%%%%%%%%%%%%%%%%%%
% Wenneker Assignment
% LaTeX Template
% Version 2.0 (12/1/2019)
%
% This template originates from:
% http://www.LaTeXTemplates.com
%
% Authors:
% Vel (vel@LaTeXTemplates.com)
% Frits Wenneker
%
% License:
% CC BY-NC-SA 3.0 (http://creativecommons.org/licenses/by-nc-sa/3.0/)
%
%%%%%%%%%%%%%%%%%%%%%%%%%%%%%%%%%%%%%%%%%

%----------------------------------------------------------------------------------------
%	PACKAGES AND OTHER DOCUMENT CONFIGURATIONS
%----------------------------------------------------------------------------------------

\documentclass[11pt]{scrartcl} % Font size
\usepackage{comment}
\usepackage{color}
%%%%%%%%%%%%%%%%%%%%%%%%%%%%%%%%%%%%%%%%%
% Wenneker Assignment
% Structure Specification File
% Version 2.0 (12/1/2019)
%
% This template originates from:
% http://www.LaTeXTemplates.com
%
% Authors:
% Vel (vel@LaTeXTemplates.com)
% Frits Wenneker
%
% License:
% CC BY-NC-SA 3.0 (http://creativecommons.org/licenses/by-nc-sa/3.0/)
% 
%%%%%%%%%%%%%%%%%%%%%%%%%%%%%%%%%%%%%%%%%

%----------------------------------------------------------------------------------------
%	PACKAGES AND OTHER DOCUMENT CONFIGURATIONS
%----------------------------------------------------------------------------------------

\usepackage{amsmath, amsfonts, amsthm} % Math packages
\usepackage{tabto}
\usepackage{listings} % Code listings, with syntax highlighting
\usepackage{tabu}
\usepackage{array}
\usepackage[english]{babel} % English language hyphenation
\usepackage{hyperref}

\usepackage{graphicx} % Required for inserting images
\graphicspath{{Figures/}{./}} % Specifies where to look for included images (trailing slash required)

\usepackage{booktabs} % Required for better horizontal rules in tables

\numberwithin{equation}{section} % Number equations within sections (i.e. 1.1, 1.2, 2.1, 2.2 instead of 1, 2, 3, 4)
\numberwithin{figure}{section} % Number figures within sections (i.e. 1.1, 1.2, 2.1, 2.2 instead of 1, 2, 3, 4)
\numberwithin{table}{section} % Number tables within sections (i.e. 1.1, 1.2, 2.1, 2.2 instead of 1, 2, 3, 4)

\setlength\parindent{0pt} % Removes all indentation from paragraphs

\usepackage{enumitem} % Required for list customisation
\setlist{noitemsep} % No spacing between list items

%----------------------------------------------------------------------------------------
%	DOCUMENT MARGINS
%----------------------------------------------------------------------------------------

\usepackage{geometry} % Required for adjusting page dimensions and margins

\geometry{
	paper=a4paper, % Paper size, change to letterpaper for US letter size
	top=2.5cm, % Top margin
	bottom=3cm, % Bottom margin
	left=3cm, % Left margin
	right=3cm, % Right margin
	headheight=0.75cm, % Header height
	footskip=1.5cm, % Space from the bottom margin to the baseline of the footer
	headsep=0.75cm, % Space from the top margin to the baseline of the header
	%showframe, % Uncomment to show how the type block is set on the page
}

%----------------------------------------------------------------------------------------
%	FONTS
%----------------------------------------------------------------------------------------

\usepackage[utf8]{inputenc} % Required for inputting international characters
\usepackage[T1]{fontenc} % Use 8-bit encoding

\usepackage{fourier} % Use the Adobe Utopia font for the document

%----------------------------------------------------------------------------------------
%	SECTION TITLES
%----------------------------------------------------------------------------------------

\usepackage{sectsty} % Allows customising section commands

\sectionfont{\vspace{6pt}\centering\normalfont\scshape} % \section{} styling
\subsectionfont{\normalfont\bfseries} % \subsection{} styling
\subsubsectionfont{\normalfont\itshape} % \subsubsection{} styling
\paragraphfont{\normalfont\scshape} % \paragraph{} styling

%----------------------------------------------------------------------------------------
%	HEADERS AND FOOTERS
%----------------------------------------------------------------------------------------

\usepackage{scrlayer-scrpage} % Required for customising headers and footers

\ohead*{} % Right header
\ihead*{} % Left header
\chead*{} % Centre header

\ofoot*{} % Right footer
\ifoot*{} % Left footer
\cfoot*{\pagemark} % Centre footer
 % Include the file specifying the document structure and custom commands
\usepackage{graphicx}
\usepackage{subcaption}



%Code retrieved from: https://www.overleaf.com/project/5c52d66b6343590b46b4fd03


%----------------------------------------------------------------------------------------
%	TITLE SECTION
%----------------------------------------------------------------------------------------

\title{
	\normalfont\normalsize
	\textsc{Old Dominion University}\\ % Your university, school and/or department name(s)
	\vspace{25pt} % Whitespace
	\rule{\linewidth}{0.5pt}\\ % Thin top horizontal rule
	\vspace{20pt} % Whitespace
	{\huge Assignment 9}\\ % The assignment title
	\vspace{12pt} % Whitespace
	\rule{\linewidth}{2pt}\\ % Thick bottom horizontal rule
	\vspace{12pt} % Whitespace
}

\author{\LARGE David Bayard} % Your name

\date{\normalsize\today} % Today's date (\today) or a custom date

\begin{document}

\definecolor{codegreen}{rgb}{0,0.6,0}
\definecolor{codegray}{rgb}{0.5,0.5,0.5}
\definecolor{codepurple}{rgb}{0.58,0,0.82}
\definecolor{backcolour}{rgb}{0.95,0.95,0.92}
\lstdefinestyle{pythonStyle}{
  backgroundcolor=\color{backcolour},
  commentstyle=\color{codegreen},
  keywordstyle=\color{magenta},
  numberstyle=\tiny\color{codegray},
  stringstyle=\color{codepurple},
  basicstyle=\footnotesize,
  breakatwhitespace=false,
  breaklines=true,
  captionpos=b,
  keepspaces=true,
  numbers=left,
  numbersep=5pt,
  showspaces=false,
  showstringspaces=false,
  showtabs=false,
  tabsize=2
}

\lstset{style=pythonStyle}


\maketitle % Print the title

\pagebreak
\section*{Question 1.}



%------------------------------------------------

\subsection*{ Use knnestimate() to compute the nearest neighbors for both:
	http://f-measure.blogspot.com/
	http://ws-dl.blogspot.com/}
\bigskip\bigskip


\large Solution:
\newline \small

\tabto{2.0cm} In order to compute the nearest neighbors for both URIs, the cosine distance metric was implemented. This was done by using the scipy library, which offers a method to calculate cosine distance as shown below. 
\newline \newline

\begin{lstlisting}[language = Python, caption = cosine distance metric]
def cosSimilarity(v1,v2):
  
  return (spatial.distance.cosine(v1, v2))
\end{lstlisting} \bigskip 

\tabto{2.0cm} This method takes two 1 dimensional arrays and calculates the distance between them as shown below.

$$ 1 - \frac{u \cdot v}{||u|| _2  \ ||v|| _2}   $$
$ where \ || * ||_2 \ is \ the \ 2 \ norm \ of \ its \ argument \ * \ and \ u   \cdot \ v \ is \ the \ cross \ product \ of \ u \ and \ v $

\bigskip
\tabto{2.0cm} In order for this method to work, two 1 dimensional arrays are required. This is done by extracting the data from the blogdata.txt file introduced in assignment 7. This file contains the frequency of 1000 words in each row, as well as the blog title. \newline \newline 

\tabto{2.0cm} To create the 1 dimensional arrays, each row must be read from the blogdata.txt file, and the blog titles need to be removed. The readfile function from assignment 7 is used to do this, returning a list of 1 dimensional lists, representing each blog. 

\begin{lstlisting}[language = Python, caption=Extract numbers from blogdata.txt file]
 
 data=[]
  for line in lines[1:]:
    p=line.strip( ).split('\t')
  # First column in each row is the rowname
    rownames.append(p[0])
  # The data for this row is the remainder of the row
    data.append([float(x) for x in p[1:]])


\end{lstlisting} \bigskip 

\tabto{2.0cm} Using the 1 dimensional list of lists, the knn estimate function may be called to find the nearest k neighbors of a specific blog, represented as a matrix. By providing this function the indexes 0 and 1 of the returned list, representing F-Measuree and Web Science and Digital Libraries Research Group respectively, the distance of k elements from those blogs is estimated. 

\begin{lstlisting}[language = Python, caption= knn measure function]
def knnestimate(data,vec1,k=5):
  # Get sorted distances
  newList = []
  dlist=getdistances(data,vec1)

  for i in range(0,k):
    newList.append(dlist[i])

  return newList
\end{lstlisting}

\tabto{2.0cm} The function above returns the nearest k neighbors of the provided matrix, against all other matrices in the blogdata.txt file. This is done by calling the getdistances function, which uses the cosSimalirity function to find the cosine distance between a pair of matrices.

\begin{lstlisting}[language = Python, caption= get distances function]
for i in range(len(data)):
    vec2=data[i]
    
    # Add the distance and the index
    distancelist.append((cosSimilarity(vec1,vec2),i))
  
  # Sort by distance
  distancelist.sort()
  return distancelist
\end{lstlisting}

\tabto{2.0cm} This function returns a sorted list of distances between every matrix in the data file and the matrix to compare. The code below depicts how to generate the nearest k neighbors of a specific matrix.


\begin{lstlisting}[language = Python, caption= get distances of F-Measure and Web Science blogs]
# Distance of k neighbors from f-measure blog
f_measureDist = helper.knnestimate(data, data[0],5)

# Distance of k neighbors from web science blog
web_ScienceDist = helper.knnestimate(data,data[1],5)
\end{lstlisting}

\tabto{2.0cm} The tables below depict the nearest neighbors for both the f-measure and web science blogs, for k = 5
\bigskip
\begin{center}
\begin{tabular}{ |p{3cm}||p{3cm}|p{5cm} }
 \hline
 \multicolumn{3}{|c|}{F-Measure Blog} \\
 \hline
 Cosine Distance & Index & Blog Title\\
 \hline
 0.0 & 0 & F-Measure\\
 0.6247023086160843 & 35 & Indie Obsessive \\
 0.7324998119138642 & 7 & 2 or 3 lines \\
 0.8257794856910945 & 24 & Dave's Music Database\\
 0.8263925840186517 & 8 & This Is Country Music\\
 \hline
\end{tabular}
\end{center}

\bigskip

\begin{center}
\begin{tabular}{ |p{3cm}||p{3cm}|p{5cm}|  }
 \hline
 \multicolumn{3}{|c|}{Web Science Blog} \\
 \hline
 Cosine Distance & Index & Blog Title\\
 \hline
 0.0 & 1 & Web Science and Digital Libraries Research Group\\
 0.40134546076893296 & 73 & BishopBlog \\
 0.4524798903145403 & 42 & Data Science Notes\\
 0.454005920449409 & 51 & Big Data Society\\
 0.48310622025301797 & 48 & The Tree of Life \\
 \hline
\end{tabular}
\end{center}

\bigskip
\tabto{2.0cm} Looking at the tables above, it is evident that the nearest neighbors are related to the blog being compares. This is due to the fact that each of the blogs closest to the one being compared maintain the same theme, either being music or data science. The data for k values of 1,2,5,10, and 20 are stored in the distanceFile.txt file.

\end{document}
